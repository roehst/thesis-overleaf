\chapter{Conclusion and Future Work}

\section{Summary of Contributions}

In this thesis, we showed how Alloy, a lightweight formal language, could be used instead to complement a data model written in JSON Schema, a data validation language.

\todo{Colocar as contribuições no check de transações, aritméticos, vesting, etc.}

\section{Limitations and Open Issues}

\todo{I should give that while Alloy is not particularly hard to use after learning, it is certainly not within the reach of any programmer \- after all it still requires a certain knowledge of set theory, relations and first order logic, beyond a basic grasp of how SAT solvers are used.}

\todo{We chose not to use temporal logic in our model, even though it is supported by Alloy since version 6 to keep the model simpler \- but it required some gadgetry to simulate the passage of time.}

\section{Future Directions for Research and Development}

\todo{Give that LLM models are becoming very important and Alloy provides a way to succintly encode models and verify data over domains with a lot of subtlety, and Alloy prompts seem to work very well with ChatGPT versions 3.5 onwards without any fine-tuning, and can be suprising with finetuning}

\todo{Alloy has a Java API that could be leveraged into a web-based editor that could bring such a useful technology much closer to developers. A SDK for embedding the Alloy Analyzer into other applications would also be very useful for open source contributors to explore code generation.}

\todo{A temporal model based on our model could be used to specify a complete server for managing captables that by construction follows a specification using event driven architectures}

\section{Conclusion}

\todo{Conclusion should be ambitious and show how exploring a domain model was much richer and ultimately leads to a better understanding of the domain, and hopefully better software for society.}

